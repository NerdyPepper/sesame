\documentclass[a4paper]{article}
\usepackage{graphicx}
\usepackage{braket}
\graphicspath{ {./images/} }

\begin{document}

% ================================================
\section{Introduction}

Grover's algorithm is a quantum algorithm that finds, with high probability,
the input to an unknown function, given the output value. It finds this value
in $ O(\sqrt{N}) $ evaluations of the function over the database.
\\
\par

Much before its invention by Lov Grover in 1996, it was proved that any quantum
algorithm would have to perform atleast $O(\sqrt{N})$ evaluations of the
function, hence, Grover's algorithm is almost optimal.
\\
\par
Like most quantum algorithms, Grover's algorithm is probablistic in nature.
That is, the probability of the algorithm producing the correct answer is less
than 1.  It \textit{is} possible to produce the correct answer all the time,
but at the cost of runtime (repeating the number of iterations improves the
probability).

\pagebreak


% ===============================================
\section{Quantum Gates used in Grover's Algorithm}

In quantum computing, the basic operations on qubits are performed by quantum gates.
They are the building blocks of quantum circuits, similar to classical logic gates.
\\
\par
Unlike classical gates, which apply mathematical reasoning to produce outputs, 
quantum gates rotate probabilities, and produce a superposition of qubits as outputs.

\subsection{ Hadamard (H) Gate }
The Hadamard gate acts on a single qubit. It is a \underline{single qubit
rotation mapping} of basis states.
\\
\par
The Hadamard gate maps the basis states $\ket{0}$ and $\ket{1}$ to 
$ \frac{\ket{0} + \ket{1}}{\sqrt2}$ and $ \frac{\ket{0} - \ket{1}}{\sqrt2}$ 
respectively.

\begin{figure}[h]
\includegraphics[width=4cm]{hadamard}
\centering
\end{figure}

\subsection{ Pauli-X Gate }
The Pauli-X gate acts on a single qubit. It is the quantum analog of the NOT gate. 
\\
\par
It can be understood as rotation of the Bloch's sphere by $\pi$ radians about the X-axis.
Since it maps $\ket{0}$ to $\ket{1}$ and vice-versa, it is also known as a
\underline{bit-flip gate}.


\begin{figure}[h]
\includegraphics[width=4cm]{paulix}
\centering
\end{figure}

\subsection{ Controlled-NOT Gate }
Controlled-NOT (CNOT) gate is an essential component to any quantum circuit. Any quantum
circuit can be expressed as a combination of CNOT gates with a certain degree of 
accuracy.
\\
\par
\pagebreak
The CNOT gate operates on 2 qubits. It flips the second input (``target") if and only 
if the first qubit (``control") is $\ket{1}$.

\begin{table}[h!]
    \centering
    \begin{tabular}{ |c|c|c|c|  }
        \hline
        \multicolumn{2}{|c}{Before} & \multicolumn{2}{|c|}{After} \\
        \hline
        Control & Target & Control & Target\\
        \hline
        $\ket{0}$ & $\ket{0}$ & $\ket{0}$ & $\ket{0}$ \\
        $\ket{0}$ & $\ket{1}$ & $\ket{0}$ & $\ket{1}$ \\
        $\ket{1}$ & $\ket{0}$ & $\ket{1}$ & $\ket{1}$ \\
        $\ket{1}$ & $\ket{1}$ & $\ket{1}$ & $\ket{0}$ \\
        \hline
    \end{tabular}
\end{table}
The behaviour of the CNOT gate appears to be very classical according to this table. 
The complexity of this gate is reflected when used in conjunction with the Hadamard
gate (used in Grover's algorithm).

\begin{figure}[h]
\includegraphics[width=4cm]{cnot}
\centering
\end{figure}

\pagebreak


% ===============================================
\section{ Unstructured Search }

Grover's algorithm is typically used to solve unstructured search problems.
Unstructured refers to databases \underline{with no order}, i.e, true random
distribution of elements. 
\\
\par

Suppose we are given a list of $N$ elements. Among these elements, there is one
element $w$, the ``winner". The problem of locating the position of this element
$w$ is called an unstructured search problem.  $$ 1, 2, 3, \cdots w \cdots N $$
\\ \par

Mathematically, in an unstructured search problem, given a set of $N$ elements,
forming a set $X = \{X_1, X_2, X_3 \ldots X_n\} $, and given a boolean function
$f: X \to \{0, 1\}$, the goal is to find an element $u$ such that $f(u) = 1$,
where $f(x)$ is given by:

\[
    f(x) = \left\{
        \begin{array}{ll}
            0 & \mbox{if } x \neq u \\
            1 & \mbox{if } x = u \\
        \end{array}
        \right.
\]

\pagebreak


% ===============================================
\section{ Grover's Method }
Consider the list of $N$ elements as proposed in \textbf{Section 3}. Before
looking at the list of items, we have no idea where the marked item is.
Therefore, any guess of its location is as good as any other. Grover's algorithm
tries to increase the probability of finding $w$ and reduce the probability of 
finding other elements, through a process knows as Amplitude Amplification.
\\
\par

To explain Amplitude Amplification classically, consider an opaque bag of $N$
balls, one of which is red. The probability of picking the red ball is
$\frac{1}{N}$.  Grover's algorithm increases the size of the red ball, making
it 10 times the size of the other balls, and reduces the size of the remaining
balls, thereby increasing the probability of finding the pink ball (the ``winner").

\end{document}
