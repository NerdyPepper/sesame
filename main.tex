\documentclass[a4paper]{article}
\usepackage{graphicx}
\usepackage{braket}
\usepackage{amsmath}
\usepackage{url}
\usepackage{hyperref}
\graphicspath{ {./images/} }

\begin{document}

\begin{titlepage}
   \begin{center}
       \vspace*{1cm}

       RV College of Engineering\\
       Bengaluru 560 059\\
       Autonomous Institution Affiliated to VTU, Belagavi

       \vspace{2.5cm}

       \includegraphics[width=0.4\textwidth]{rv}
       \vspace{0.5cm}

       \textbf{Grover's Algorithm}

       \vspace{0.5cm}
       A Quantum Algorithm by Lov Grover

       \vfill


       \textbf{Akshay Oppiliappan (1RV18CS016)}

       \textbf{Anish Kuila (1RV18CS023)}

       \vspace{1.5cm}

       Submitted to\\
       Prof. Tribikram Gupta \\
       Dept. of Physics, RVCE

       \vspace{0.8cm}



   \end{center}
\end{titlepage}
\pagebreak

\begin{titlepage}
   \begin{center}
       \vspace*{1cm}

       RV College of Engineering\\
       Bengaluru 560 059\\
       Autonomous Institution Affiliated to VTU, Belagavi

       \vspace{2.5cm}

       \includegraphics[width=0.4\textwidth]{rv}
       \vspace{0.5cm}

       \textbf{Certificate}
   \end{center}

       \vspace{0.5cm}

       Certified that the Assignment topic ``Grover’s Algorithm" is carried out
       by Akshay (Roll No. 12) and Anish Kuila (Roll No. 19) who are bonafide
       students of R V College of Engineering, Bengaluru in partial fulfilment
       of the award of assignment marks for the I/II semester academic year
       2018 --- 19 in Engineering Physics Course. It is certified that all
       corrections/suggestions indicated for the internal assessment have been
       incorporated in the report, and a soft copy is deposited in the
       department library. The assignment report has been approved as it
       satisfies the academic requirement in respect of the work prescribed by
       the institution for the said course.

       \vspace{1.5cm}

       \begin{table}[h!]
           \begin{tabular}{ |c|c| }
               \hline
               Max. Marks & 20\\
               \hline
               Marks. Obtd. & ~~~\\
               \hline
           \end{tabular}
       \end{table}

       \vspace{0.8cm}

\end{titlepage}

\tableofcontents
\pagebreak

% ================================================
\section{Introduction}

Grover's algorithm is a quantum algorithm that finds, with high probability,
the input to an unknown function, given the output value. It finds this value
in $ O(\sqrt{N}) $ evaluations of the function over the database.
\\
\par

Much before its invention by Lov Grover in 1996, it was proved that any quantum
algorithm would have to perform atleast $O(\sqrt{N})$ evaluations of the
function, hence, Grover's algorithm is almost optimal.
\\
\par
Like most quantum algorithms, Grover's algorithm is probablistic in nature.
That is, the probability of the algorithm producing the correct answer is less
than 1.  It \textit{is} possible to produce the correct answer all the time,
but at the cost of runtime (repeating the number of iterations improves the
probability).
\\
\par

Problems which are undecidable using classical computers remain undecidable
using quantum computers. What makes quantum algorithms interesting is
that they might be able to solve some problems faster than classical algorithms
because the quantum superposition and quantum entanglement that quantum
algorithms exploit probably can't be efficiently simulated on classical
computers, known as Quantum supremacy.
\\
\par

The most well known algorithms are Shor's algorithm for factoring, and Grover's
algorithm for searching an unstructured database or an unordered list. Shor's
algorithms runs much (almost exponentially) faster than the best known
classical algorithm for factoring, the general number field sieve.Grover's
algorithm runs quadratically faster than the best possible classical algorithm
for the same task, a linear search

\pagebreak


% ===============================================
\section{Quantum Gates used in Grover's Algorithm}

In quantum computing, the basic operations on qubits are performed by quantum gates.
They are the building blocks of quantum circuits, similar to classical logic gates.
\\
\par
Unlike classical gates, which apply mathematical reasoning to produce outputs, 
quantum gates rotate probabilities, and produce a superposition of qubits as outputs.

\subsection{ Hadamard (H) Gate }
The Hadamard gate acts on a single qubit. It is a \underline{single qubit
rotation mapping} of basis states.
\\
\par
The Hadamard gate maps the basis states $\ket{0}$ and $\ket{1}$ to 
$ \frac{\ket{0} + \ket{1}}{\sqrt2}$ and $ \frac{\ket{0} - \ket{1}}{\sqrt2}$ 
respectively.

\begin{figure}[h]
\includegraphics[width=4cm]{hadamard}
\centering
\end{figure}

\subsection{ Pauli-X Gate }
The Pauli-X gate acts on a single qubit. It is the quantum analog of the NOT gate. 
\\
\par
It can be understood as rotation of the Bloch's sphere by $\pi$ radians about the X-axis.
Since it maps $\ket{0}$ to $\ket{1}$ and vice-versa, it is also known as a
\underline{bit-flip gate}.


\begin{figure}[h]
\includegraphics[width=4cm]{paulix}
\centering
\end{figure}

\subsection{ Controlled-NOT Gate }
Controlled-NOT (CNOT) gate is an essential component to any quantum circuit. Any quantum
circuit can be expressed as a combination of CNOT gates with a certain degree of 
accuracy.
\\
\par
\pagebreak
The CNOT gate operates on 2 qubits. It flips the second input (``target") if and only 
if the first qubit (``control") is $\ket{1}$.

\begin{table}[h!]
    \centering
    \begin{tabular}{ |c|c|c|c|  }
        \hline
        \multicolumn{2}{|c}{Before} & \multicolumn{2}{|c|}{After} \\
        \hline
        Control & Target & Control & Target\\
        \hline
        $\ket{0}$ & $\ket{0}$ & $\ket{0}$ & $\ket{0}$ \\
        $\ket{0}$ & $\ket{1}$ & $\ket{0}$ & $\ket{1}$ \\
        $\ket{1}$ & $\ket{0}$ & $\ket{1}$ & $\ket{1}$ \\
        $\ket{1}$ & $\ket{1}$ & $\ket{1}$ & $\ket{0}$ \\
        \hline
    \end{tabular}
\end{table}
The behaviour of the CNOT gate appears to be very classical according to this table. 
The complexity of this gate is reflected when used in conjunction with the Hadamard
gate (used in Grover's algorithm).

\begin{figure}[h]
\includegraphics[width=4cm]{cnot}
\centering
\end{figure}

\pagebreak


% ===============================================
\section{ Unstructured Search }

~~~~The phrase unstructured data usually refers to information that doesn't reside
in a traditional row-column database. There is no ``predefined" data model to 
represent unstructured data.
\\
\par

Unstructured data often includes text and multimedia content. Examples
include e-mail messages, word processing documents, videos, photos, audio
files, presentations, webpages and many other kinds of business documents. Note
that while these sorts of files may have an internal structure, they are still
considered ``unstructured" because the data they contain doesn't fit neatly in a
database. It is very difficult to analyze and infer unstructured data.
\\
\par

Grover's algorithm is typically used to solve unstructured search problems.
Unstructured refers to databases \underline{with no defined data model}, i.e,
true random distribution of elements. 
\\
\par

Suppose we are given a list of $N$ elements. Among these elements, there is one
element $w$, the ``winner". The problem of locating the position of this element
$w$ is called an unstructured search problem.  $$ 1, 2, 3, \cdots w \cdots N $$
\\ \par

Mathematically, in an unstructured search problem, given a set of $N$ elements,
forming a set $X = \{X_1, X_2, X_3 \ldots X_n\} $, and given a boolean function
$f: X \to \{0, 1\}$, the goal is to find an element $u$ such that $f(u) = 1$,
where $f(x)$ is given by:

\[
    f(x) = \left\{
        \begin{array}{ll}
            0 & \mbox{if } x \neq u \\
            1 & \mbox{if } x = u \\
        \end{array}
        \right.
\]

\pagebreak


% ===============================================
\section{ Grover's Method }
Consider the list of $N$ elements as proposed in \textbf{Section 3}. Before
looking at the list of items, we have no idea where the marked item is.
Therefore, any guess of its location is as good as any other. Grover's algorithm
tries to increase the probability of finding $w$ and reduce the probability of 
finding other elements, through a process knows as Amplitude Amplification.
\\
\par

To explain Amplitude Amplification classically, consider an opaque bag of $N$
balls, one of which is red. The probability of picking the red ball is
$\frac{1}{N}$.  Grover's algorithm increases the size of the red ball, making
it 10 times the size of the other balls, and reduces the size of the remaining
balls, thereby increasing the probability of finding the red ball (the ``winner").
\\
\par

\subsection{ State Setup }
We need to set up a quantum state, which we will use as our search space. It should 
contain an unstructured set of elements, each with equal probability of occurence. 
\\
\par
This can be prepared as a superposition of qubits as follows:

$$ \ket{\psi} = H^{n}\ket{0}^{n} $$

Thus, a quantum state prepared using 3 qubits:

\[
    H^{3}\ket{000} = \frac{1}{2\sqrt{2}}\ket{000} +
    \frac{1}{2\sqrt{2}}\ket{001} + \cdots + \frac{1}{2\sqrt{2}}\ket{111} 
\]

\begin{figure}[h]
\includegraphics[width=10cm]{init_amp}
\centering
\end{figure}

After the state preparation, the Grover’s algorithm turns into an iterative
process which composes of multiple iterations of Oracle function and the Grover
operator.
\pagebreak

\subsection{ The Oracle }
The Oracle function encodes the function $f(x)$ into the list of $N$ items, where
$f$ returns $f(x) = 0$ for all unmarked items and $f(u) = 1$ where $u$ is the 
``winner". The Oracle transformation can be defined for a given state $\ket{x}$ as:

$$ U_f\ket{x} = (-1)^{f(x)}\ket{x} $$
\par

Any $\ket{x}$ with $f(x) = 0$ is not modified, whereas any $\ket{x}$ with $f(x) = 1$
is flipped to have negative amplitude.

\begin{figure}[h]
\includegraphics[width=10cm]{flipped_amp}
\centering
\end{figure}

\subsection{ Grover's Operator }
The second and last step of Grover's iterative process is applying Grover's operator:

$$ G = (2\ket{\psi}\bra{\psi} - I)O $$

Grover's operator flips every amplitude around the average line:

$$ (2\ket{\psi}\bra{\psi} - I)\sum{a_i \ket{i}} = \sum{(2\braket{a} - a_i)\ket{i}} $$

So after applying Grover's operator, the superposition becomes:

\begin{figure}[h]
\includegraphics[width=10cm]{reflip_increased}
\centering
\end{figure}

So, if we make a measurement of the quantum state, the probability of it
collapsing into the ``winner" is greater. If we measure the qubits now, we
still have a reasonable chance of selecting the wrong answer. So we repeat this
process square root times $\sqrt{N}$ to amplify the amplitude of the right answer.

\pagebreak

For example, if we repeat the Oracle and the Grover operator once more, the
amplitude of the correct answer will stand out more

\begin{figure}[h]
\includegraphics[width=10cm]{reflip_2}
\centering
\end{figure}

% ================================================
\section{ A More Mathematical Approach }

After setting up the quantum state and applying the Oracle function, the 
superposition is:

\begin{figure}[h]
\includegraphics[width=10cm]{flipped_amp}
\centering
\end{figure}

\par

The original quantum state $\ket{\psi}$ is:

$$ \ket{\psi} = \frac{1}{2\sqrt{2}}\sum{\ket{x}} $$

\par

The state after the Grover operation becomes:

\begin{align*}
    & (2\ket{\psi}\bra{\psi} - I)\ket{x} \\
    = & (2\ket{\psi}\bra{\psi} - I) \left[ \ket{\psi} - \frac{2}{2\sqrt{2}}\ket{011} \right] \\
    = & 2\ket{\psi}\braket{\psi|\psi} - \ket{\psi} - \frac{2}{\sqrt{2}}\ket{\psi}\braket{\psi|011} + \frac{1}{\sqrt{2}}\ket{011}
\end{align*}

On substituting the inner product, $\braket{\psi|011} = \braket{011|\psi} = \frac{1}{2\sqrt{2}}$:

\begin{align*}
    & 2\ket{\psi}\braket{\psi|\psi} - \ket{\psi} - \frac{2}{\sqrt{2}}\ket{\psi}\braket{\psi|011} + \frac{1}{\sqrt{2}}\ket{011} \\
    = & 2\ket{\psi} - \ket{\psi} - \frac{2}{\sqrt{2}}\left(\frac{1}{2\sqrt{2}}\right)\ket{\psi} + \frac{1}{\sqrt{2}}\ket{011} \\
    = & \frac{1}{2}\ket{\psi} + \frac{1}{\sqrt{2}}\ket{011}
\end{align*}

Replacing $\ket{\psi}$ as the superposition, we get the final state as:

$$ = \frac{1}{4\sqrt{2}}\sum_{x \neq 3}{\ket{x}} + \frac{5}{4\sqrt{2}}\ket{011} $$

Which is the result after the first iteration:

\begin{figure}[h]
\includegraphics[width=10cm]{reflip_increased}
\centering
\end{figure}

\pagebreak

% ================================================
\section{ Time Complexity of Grover's Algorithm }

Like most quantum algorithms, Grover's Algorithm is a probablistic algorithm.
The probablity of recieving the correct answer is less than 1. After
approximately $\sqrt{N}$ cycles of Oracle and Grover operator, the probablity
of finding the correct element reaches 1 approximately.
\\
\par
The analogous search problem in classical computing cannot be solved in fewer
than $O(N)$ evaluations, because, in the worst case, the $N^{th}$ element could
be the correct element.
\\
\par
\subsection{Hidden Variable Method}
~~~It has been theorized that a non-local hidden variable quantum computer could implement
a search in an $N$-item database in approximately $O(\sqrt[3]{N})$ evaluations. Hidden
variable theories suggest that quantum mechanics do not give a complete description
of a quantum state, i.e., quantum mechanics is an \underline{incomplete theory}. A complete
theory to explain the physics of the particles we know, would eradicate all indeterminism.

\subsection{Optimality of Grover's Algorithm}
Grover's algorithm is optimal. Any algorithm that accesses a database using an Oracle
function would \textit{have to} apply it as many times as Grover's algorithm.
\\
\par
Grover's algorithm suggests that all NP(non-deterministic polynomial time) 
complete problems fall under BQP(bounded error quantum polynomial) problems, because
all NP complete problems could potentially be converted into Grover-type search problems,
but it does not prove the relation.

\pagebreak


% ================================================
\section{ Applications }
~~~Grover's algorithm, in general, can be described as an algorithm to invert a function.
In other words, if $y = f(x)$ can be evaluated on a quantum computer, Grover's algorithm
computes $x$ for a given $y$.
\\
The applications of Grover's algorithm include (but are not limited to):
\begin{enumerate}
        \item Estimating the mean of a database
        \item Estimating the median of a database 
        \item Break cryptographic hash functions 
        \item Searching databases
\end{enumerate}

\section { Limitations }
~~~The comparision of classical to quantum algorithms make some rather bold assumptions,
namely:

\begin{enumerate}
        \item The database is expressed in the form of a superposition of qubits
        \item Black-boxed Oracle function. It is assumed that the Oracle function
            is computable on a quantum computer
\end{enumerate}
\par
Further, the separation of Oracle function (to evaluate the constraint) from the actual
search algorithm, prevents optimiztion. Classical algorithms take advantage of the relation
between the constraint function and the search algorithm to prevent exhaustive search.
\par

In the near future, Grover's algorithm running on a quantum computer cannot compete
with the best classical computers in practical applications, such as a Web search, because
Grover's algorithm cannot take advantage of structure present in data, if any.

\pagebreak


% ================================================
\section{ Bibliography }

Most of the information in this document has been borrowed from the following 
sources:

\begin{enumerate}
    \item \href{https://medium.com/@jonathan_hui/qc-grovers-algorithm-cd81e61cf248}{Grover's Search by Jonathan Hui}
    \item \href{https://web.eecs.umich.edu/~imarkov/pubs/jour/cise05-grov.pdf}{'Is Quantum Search Practical' by John P. Hayes}
    \item \href{https://www.scottaaronson.com/papers/qchvpra.pdf}{'Quantum computing and Hidden Variables' by Scott Aaronson}
    \item \href{https://quantumexperience.ng.bluemix.net/proxy/tutorial/full-user-guide/004-Quantum_Algorithms/}{'Grovers Algorithm' by IBM Q}
\end{enumerate}

This document was written in \LaTeX ~and can be compiled using the texlive
package. The source of this document is completely free and open source.
Readers are welcome to modify and share it as they please.  It is available at
\url{https://github.com/nerdypepper/sesame}.

\end{document}
