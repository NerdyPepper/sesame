\documentclass[a4paper]{article}
\usepackage{graphicx}
\usepackage{braket}
\graphicspath{ {./images/} }

\begin{document}

% ================================================
\section{Introduction}

Grover's algorithm is a quantum algorithm that finds, with high probability,
the input to an unknown function, given the output value. It finds this value
in $ O(\sqrt{N}) $ evaluations of the function over the database.
\\
\par
If was invented by Lov Grover in the year 1996. It was proved in 1997, by
Bennet Bernstein, that any quantum search algorithm has to perform atleast $
O(\sqrt{N}) $ of the function, hence, Grover's algorithm is almost optimal.
\\
\par
Like most quantum algorithms, Grover's algorithm is probablistic in nature.
That is, the probability of the algorithm producing the correct answer is less
than 1.  It \textit{is} possible to produce the correct answer all the time,
but at the cost of runtime (repeating the number of iterations improves the
probability).

\pagebreak


% ===============================================
\section{Quantum Gates used in Grover's Algorithm}

In quantum computing, the basic operations on qubits are performed by quantum gates.
They are the building blocks of quantum circuits, similar to classical logic gates.
\\
\par
Unlike classical gates, which apply mathematical reasoning to produce outputs, 
quantum gates rotate probabilities, and produce a superposition of qubits as outputs.

\subsection{Hadamard (H) gate}

The Hadamard gate acts on a single qubit. It is a \underline{single qubit
rotation mapping} of basis states.
\\
\par
The Hadamard gate maps the basis states $\ket{0}$ and $\ket{1}$ to 
$ \frac{\ket{0} + \ket{1}}{\sqrt2}$ and $ \frac{\ket{0} - \ket{1}}{\sqrt2}$ 
respectively.

\begin{figure}[h]
\includegraphics[width=4cm]{hadamard}
\centering
\end{figure}

\subsection{ Pauli-X Gate }

The Pauli-X gate acts on a single qubit. 

\end{document}
